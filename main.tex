%!TeX program = xelatex

%% start of file `main.tex'.
%% Copyright 2014 Francois Mouton (moutonf@gmail.com).
%
% This template is adapted from the work performed by Xavier Danaux (xdanaux@gmail.com).
% This template further extends the functionality by integrating the moderntimeline package.
% This template also includes custom Biblatex style to print bibliography items with the moderntimeline package.
%
% This work may be distributed and/or modified under the
% conditions of the LaTeX Project Public License version 1.3c,
% available at http://www.latex-project.org/lppl/.


\documentclass[10pt,a4paper,sans]{moderncv}        % possible options include font size ('10pt', '11pt' and '12pt'), paper size ('a4paper', 'letterpaper', 'a5paper', 'legalpaper', 'executivepaper' and 'landscape') and font family ('sans' and 'roman')

% moderncv themes
\moderncvstyle{classic}                             % Only the 'classic' style is fully functional with the modifications made. The other options, 'casual' (default), 'oldstyle' and 'banking' has minor typesetting problems with the current modifications.
%\moderncvcolor{purple}                               % color options 'blue' (default), 'orange', 'green', 'red', 'purple', 'grey' and 'black'

\definecolor{color0}{rgb}{0,0,0}% black
\definecolor{color1}{rgb}{0.00,0.38,0.43}% dark teal
\definecolor{color2}{rgb}{0.45,0.45,0.45}% dark grey
\definecolor{color4}{rgb}{0.04,0.12,0.50}% dark blue
% \definecolor{link}{rgb}{0.00,0.38,0.43}

%\renewcommand{\familydefault}{\sfdefault}         % to set the default font; use '\sfdefault' for the default sans serif font, '\rmdefault' for the default roman one, or any tex font name

%------------------Compile with xelatex!--------------------------------
\usepackage{fontspec} % For loading fonts
\defaultfontfeatures{Mapping=tex-text}
\setsansfont[]{Source Sans Pro}
\setmonofont[]{Inconsolata} % Mono font

%-------------------Mess with link colors ------------------------------
\AtBeginDocument{%
  \hypersetup{%
    colorlinks = true,%
    linkcolor  = color4,%
    urlcolor   = color4,%
    allbordercolors={0 0 0},%
    pdfborderstyle={/S/U/W 1},%
  }%
  
}
%\renewcommand{\familydefault}{\sfdefault}

% character encoding
% \usepackage[utf8]{inputenc}                       % if you are not using xelatex ou lualatex, replace by the encoding you are using

% adjust the page margins
\usepackage[scale=0.78]{geometry}
%\setlength{\hintscolumnwidth}{3cm}                % if you want to change the width of the column of the timeline
%\setlength{\makecvtitlenamewidth}{10cm}           % for the 'classic' style, if you want to force the width allocated to your name and avoid line breaks. Be careful though, the length is normally calculated to avoid any overlap with your personal info; use this at your own typographical risks.

%-------------------Inlcuding pdfpages package-------------------------------------------------------------

\usepackage{pdfpages}

%-------------------Including moderntimeline package-------------------------------------------------------

\usepackage{moderntimeline}

\tlmaxdates{2007}{2023}                             % Set the scale of the timeline. \tlmaxdates{startDate}{endDate}

%-------------------Including xpatch package---------------------------------------------------------------

\usepackage{xpatch}

%-------------------Including Biblatex package-------------------------------------------------------------

\usepackage[url=false,
    backend=bibtex,                                  % This can be set to either biber or bibtex. If references are missing just change back and forth between biber and bibtex..
    style=authoryear,
    doi=false,  
    isbn=false,
    backref=false,
    dashed=false,                                   % Do not add a dash out authors for subsequent articles with the same authors.
    maxnames=99,                                    % Amount of authors to include before abbreviating.
    sorting=ydnt]{biblatex}                         % Sorting in reverse order

\addbibresource{cvreferences.bib}                   % Include your bibtex file here. Format: fileName.bib

\input{biblatex_modifications/standard_modification.tex}        % Modifying the default standard.tex style of Biblatex. This modification is performed to include the moderntimeline package.

%-------------------Patching stuff from old moderncv-----------------------------------------------------
\newlength{\listdoubleitemmaincolumnwidth}%
\setlength{\listdoubleitemmaincolumnwidth}{6.5cm}% 

%-------------------Defining a CV Reference column style and a CV reference entry block-------------------

% Adapted from the solution provided in: http://tex.stackexchange.com/questions/34881/references-section-in-a-cv
% usage: \cvreference{name}{address line 1}{address line 2}{address line 3}{address line 4}{e-mail address}{phone number}{mobile phone number}
% Everything but the name is optional
% If \addresssymbol, \emailsymbol or \phonesymbol are specified, they will be used.
% (Per default, \addresssymbol isn't specified, the other two are specified.)
% If you don't like the symbols, remove them from the following code, including the tilde ~ (e.g. \phonesymbol~).

\newcommand{\cvreferencecolumn}[2]{%
  \cvitem[0.75em]{}{%
    \begin{minipage}[t]{\listdoubleitemmaincolumnwidth}#1\end{minipage}%
    \hfill%
    \begin{minipage}[t]{\listdoubleitemmaincolumnwidth}#2\end{minipage}%
    }%
}

\newcommand{\cvreference}[8]{%
    \textbf{#1}\newline% Name
    \ifthenelse{\equal{#2}{}}{}{\addresssymbol~#2\newline}%
    \ifthenelse{\equal{#3}{}}{}{#3\newline}%
    \ifthenelse{\equal{#4}{}}{}{#4\newline}%
    \ifthenelse{\equal{#5}{}}{}{#5\newline}%
    \ifthenelse{\equal{#6}{}}{}{\emailsymbol~\texttt{\href{mailto:#6}{\nolinkurl{#6}}}\newline}%
    \ifthenelse{\equal{#7}{}}{}{\phonesymbol~#7\newline}
    \ifthenelse{\equal{#8}{}}{}{\mobilephonesymbol~#8}}

    \usepackage{tikz}

\newcommand{\ExternalLink}{%
    \tikz[x=1.2ex, y=1.2ex, baseline=-0.05ex]{% 
        \begin{scope}[x=1ex, y=1ex]
            \clip (-0.1,-0.1) 
                --++ (-0, 1.2) 
                --++ (0.6, 0) 
                --++ (0, -0.6) 
                --++ (0.6, 0) 
                --++ (0, -1);
            \path[draw, 
                line width = 0.5, 
                rounded corners=0.5] 
                (0,0) rectangle (1,1);
        \end{scope}
        \path[draw, line width = 0.5] (0.5, 0.5) 
            -- (1, 1);
        \path[draw, line width = 0.5] (0.6, 1) 
            -- (1, 1) -- (1, 0.6);
        }
    }
    
%-------------------Personal Data for CV title-----------------------------------------------------------
% Example:
\name{David}{Charte}
\title{Data scientist}                               % optional, remove / comment the line if not wanted
\address{}{}{Granada, Spain}% optional, remove / comment the line if not wanted; the "postcode city" and and "country" arguments can be omitted or provided empty
\phone[mobile]{+34~697~494~169}                   % optional, remove / comment the line if not wanted
%\phone[fixed]{+2~(345)~678~901}                    % optional, remove / comment the line if not wanted
%\phone[fax]{+3~(456)~789~012}                      % optional, remove / comment the line if not wanted
\email{david@deivi.ch}                               % optional, remove / comment the line if not wanted
\homepage{deivi.ch}                         % optional, remove / comment the line if not wanted
\extrainfo{born 19 oct 1994}                 % optional, remove / comment the line if not wanted
\photo[48pt][0.2pt]{avatar}                       % optional, remove / comment the line if not wanted; '64pt' is the height the picture must be resized to, 0.4pt is the thickness of the frame around it (put it to 0pt for no frame) and 'picture' is the name of the picture file stored
% \github{fdavidcl}
%\quote{Quote}                                 % optional, remove / comment the line if not wanted


%-------------------------------------------------------------------------------------------------------
%   Content
%-------------------------------------------------------------------------------------------------------
\begin{document}

%\social[linkedin]{fdavidcl}
\social[github]{fdavidcl}
%-------------------Resume------------------------------------------------------------------------------

\makecvtitle

\vspace{-2em}
\section{Experience}

%% %-------------------Vocational Experience---------------------------------------------------------------

%% \subsection{Vocational}

%% % Format: \tlcventry{StartYear}{EndYear}{Job title}{Employer}{City}{Country (optional)}{General description no longer than 1--2 lines.\newline{}%
%% % Example:
%% % \tlcventry{2008}{2011}{System Administrator}{Simple Solutions}{MyCity}{}{Did system administrative work.\newline{}%
%% % Main Duties:%
%% %  \begin{itemize}%
%% %      \item Administrate the servers;
%% %      \item Administrate employee computers 
%% %          \begin{itemize}%
%% %              \item All employee's computers had to be up to date;
%% %          \end{itemize}
%% %      \item Did some more administrating
%% %   \end{itemize}}

\tlcventry{2018}{0}{Doctoral student}{Universidad de Granada}{}{supervisors: Francisco Herrera \& Francisco Charte}{Lecturer training contract under Spanish FPU program ref. FPU17/04069, defending thesis soon.\begin{itemize}\item \textbf{Main topic}: \emph{Finding alternative representations for data through deep learning techniques}\item Collaborated with \textbf{Repsol} on optimization of refinery processes (machine learning and autoencoders)\item Collaborated with \textbf{ArcelorMittal} on \href{https://github.com/ari-dasci/S-metallograph-segmentation}{semantic segmentation of metallographic microstructures \ExternalLink}\newline (adaptation and training of semi-supervised fully convolutional models)\item Directed two bachelor's theses on automatic melody synthesis with autoencoders and neural search for COVID-19 detection in chest X-rays, respectively\item Published \href{https://www.youtube.com/playlist?list=PL88MWrW4s4nf-Bc3hccxt3Att8TSS-LBn}{a 5-part free online course \ExternalLink} (in Spanish) on linear algebra and dimensionality reduction\end{itemize}
}
\tlcventry{2020}{0}{Online course author and lecturer}{CampusMVP}{Spain}{}{Course on Data Science and Machine Learning with both video-based and written lectures
}
\tldatecventry{2018}{Researcher}{Universidad de Granada}{}{supervisor: Francisco Herrera}{Research contract with project \emph{BigDaPTOOLS}. Task: \emph{Development of data preprocessing libraries in R}
}
\tlcventry{2016}{2018}{Undergraduate researcher}{Universidad de Granada}{}{}{Research grant.
Topic: \emph{Interpretative analysis of unsupervised deep learning techniques and extraction of multi-view models for supervised learning}
}
%-------------------Education Section-------------------------------------------------------------------

\section{Education}

% For a date range: (To indicate 'up to present', set EndYear to 0)
% Format:  \tlcventry{StartYear}{EndYear}{Degree}{Institution}{City}{\textit{Grade}}{Description}  % Arguments 3 (Degree) to 6 (Grade) can be left empty. 
% Example: \tlcventry{2012}{0}{BSc Computer Science}{University of MyCity}{MyCity}{}{Also completed several random courses}

%\tlcventry{2018}{0}{PhD in Computer Science}{Universidad de Granada}{Granada}{}{}

\tlcventry{2017}{2018}{M.Sc. in Data Science and Computer Engineering}{Universidad de Granada}{Granada}{}{Emphasis in data science}

\tlcventry{2012}{2017}{B.Sc. in Computer Science}{Universidad de Granada}{Granada}{\textit{9.40/10}}{}

\tlcventry{2012}{2017}{B.Sc. in Mathematics}{Universidad de Granada}{Granada}{\textit{9.04/10}}{}

% For a single year:
% Format:  \tldatecventry{StartYear}{Degree}{Institution}{City}{\textit{Grade}}{Description}
% Example: \tldatecventry{2008}{Senior Certificate}{High School MyCity}{MyCity}{\textit{80\%}}{Passed with distinction}

\tldatecventry{2017}{Intl. Summer School on Deep Learning}{Universidad de Deusto \& Rovira i Virgili University}{}{}{}

\tldatecventry{2014}{A practical approach to Data Science and Big Data}{Intl. University of Andalusia (UNIA)}{}{}{}

\tldatecventry{2013}{New Trends on Computer Engineering}{Centro Mediterráneo (UGR)}{}{}{}


\tlcventry{2007}{2012}{Project for detection/stimulus of mathematical talent (ESTALMAT)}{SAEM-Thales}{Granada}{}{}

%-------------------Achievements Section----------------------------------------------------------------

%% \section{Achievements}

%% % Format:  \cvlistitem{Achievement}
%% % Example: \cvlistitem{Received best student award}
%% % Example: \cvlistitem{Another achievement. This achievement is particularly long and therefore normally spans over several lines. Did you notice the indentation when the line wraps?}

%% \cvlistitem{Received best student award}
%% \cvlistitem{Another achievement. This achievement is particularly long and therefore normally spans over several lines. Did you notice the indentation when the line wraps?}

%-------------------Languages Section-------------------------------------------------------------------

% \section{Languages}

% Format:  \cvitemwithcomment{Language}{Skill level}{Comment}
% Example: \cvitemwithcomment{English}{Native}{Mother Tongue}
% Example: \cvitemwithcomment{French}{Fluent}{Daily practice, all work performed in English}

% \cvitemwithcomment{Spanish}{Native}{}
% \cvitemwithcomment{English}{C1 level}{Cambridge CAE grade A: C1 level with C2 abilities}
% \cvitemwithcomment{French}{B2 level}{DELF B2 certificate}
% \cvitemwithcomment{Swedish}{A2 level}{}

%-------------------Skills Matrix Section----------------------------------------------------------------

\section{Skills}

% For items with categories: 
% Format:  \cvdoubleitem{Category}{List of skills}{Category Name}{List of skills}
% Note: It looks better if the category is bold with \textbf{}
% Example:
% \subsection{Development}
% \cvdoubleitem{\textbf{Languages}}{C\#, C\+\+, Java}{\textbf{Databases}}{MSSQL, MySQL}
%
% For a bullet list without categories:
% Format:  \cvlistdoubleitem{Skill 1}{Skill 2}
% Example: 
% \subsection{Development}
% \cvlistdoubleitem{C\#, Java, Ruby}{MSSQL, MySQL}
% \cvlistdoubleitem{Photoshop}{Windows, Linux. In the single column list, this item is particularly long to wrap over several lines.}

\cvitem{\textbf{Soft skills}}{Learns fast $\cdot$ Loves teaching $\cdot$ Team player $\cdot$ Natural problem solver $\cdot$ Organized and meticulous}
\cvitem{\textbf{Languages}}{Spanish (native)~$\cdot$~English (Advanced, CEFR C1)~$\cdot$~French (Intermediate, CEFR B2)~$\cdot$~Swedish (Basic)}
\subsection{Data Science}
\cvdoubleitem{\textbf{Models}}{Autoencoders $\cdot$ (Fully) Convolutional Networks $\cdot$ Standard machine learning}{\textbf{Technologies}}{Tensorflow/Keras $\cdot$ Pytorch $\cdot$ Scikit-Learn $\cdot$ Matplotlib}


\subsection{Development}
\cvdoubleitem{\textbf{Programming}}{Python $\cdot$ R $\cdot$ Ruby $\cdot$ C++ $\cdot$ C $\cdot$ Shell $\cdot$ SQL}{\textbf{Web}}{HTML $\cdot$ JavaScript $\cdot$ CSS $\cdot$ RWD $\cdot$ Vue.js}
             

\subsection{Misc}
\cvdoubleitem{\textbf{Systems}}{Linux/UNIX $\cdot$ Docker $\cdot$ NGINX}{\textbf{Tools}}{Git $\cdot$ GitHub $\cdot$ \LaTeX $\cdot$ Emacs}

%\subsection{Utilities}
%\cvdoubleitem{\textbf{Documents}}{\LaTeX{}, LibreOffice}{\textbf{Graphics}}{Inkscape, Krita}
             
%% \subsection{Otras}
%% \cvlistdoubleitem{Fast learner}{Good mediator}

%-------------------Experience Section------------------------------------------------------------------




\clearpage
%-------------------Publications Section----------------------------------------------------------------
% The cvitem commands needs to be altered to correctly print all publications with the moderntime package.
% The cvitem command is edited to remove all forced punctuation within the command.
% All the typesetting of the text is handled by the modified Biblatex style.

\input{cvitem_modifications/cvitem_modified}        % Removing forced punctuation from cvitem

\nocite{*}                                          % Print all publications.

% Format:  \printbibliography[type=Biblatex type,title={Title of publication}]
% Example: \printbibliography[type=article,title={Journal Publications}]
% Example: \printbibliography[type=inproceedings,title={Conference Publications}]
% Example: \printbibliography[type=thesis,title={Thesis}]

\printbibliography[type=article,title={Journal Publications}]
%\printbibliography[type=inproceedings,title={Conference Publications}]
%\printbibliography[type=thesis,title={Thesis}]

\vspace{-.5em}
\cvitem{\textbf{Submitted works}}{David Charte, Francisco Charte and Francisco Herrera. ``Reducing Data Complexity using Autoencoders with Class-informed Loss Functions''}
\cvitem{}{Julián Luengo, Raúl Moreno, Iván Sevillano, David Charte, Adrián Peláez-Vegas, et al. ``A tutorial on the segmentation of metallographic images: taxonomy, new MetalDAM dataset, deep learning-based ensemble model, experimental analysis and challenges''}

\input{cvitem_modifications/cvitem_moderncvclassic} % Reverting changes to cvitem.

%-------------------PhD Thesis Section------------------------------------------------------------------

%\section{Bachelor's Thesis}

% Format:  \cvitem{Section Name}{Description}
% Example: \cvitem{title}{\emph{The title of my PhD goes here}}
% Example: \cvitem{supervisors}{My supervisors' names go here}
% Example: \cvitem{description}{Short thesis abstract}

%\cvitem{Title}{\href{http://dx.doi.org/10.13140/RG.2.2.16155.57123/1}{\emph{Reducción de la dimensionalidad en problemas de clasificación con Deep Learning}}}
%\cvitem{Supervisor}{Francisco Herrera Triguero}
%\cvitem{Description}{Theoretical analysis of techniques based on deep neural networks that tackle the dimensionality reduction problem. Software for usage of these models and visualization generation.}
%\cvitem{Grade}{10/10 (Distinction)}

%-------------------Masters Thesis Section--------------------------------------------------------------

%% \section{Master thesis}

%% % Format:  \cvitem{Section Name}{Description}
%% % Example: \cvitem{title}{\emph{The title of my Masters goes here}}
%% % Example: \cvitem{supervisors}{My supervisors' names go here}
%% % Example: \cvitem{description}{Short thesis abstract}

%% \cvitem{title}{\emph{The title of my Masters goes here}}
%% \cvitem{supervisors}{My supervisors' names go here}
%% \cvitem{description}{Short thesis abstract}


%-------------------Projects---------------------------------------------------------------

\section{Projects}

%\subsection{Vocational}

% Format: \tlcventry{StartYear}{EndYear}{Job title}{Employer}{City}{Country (optional)}{General description no longer than 1--2 lines.\newline{}%
% Example:
% \tlcventry{2008}{2011}{System Administrator}{Simple Solutions}{MyCity}{}{Did system administrative work.\newline{}%
% Main Duties:%
%  \begin{itemize}%
%      \item Administrate the servers;
%      \item Administrate employee computers 
%          \begin{itemize}%
%              \item All employee's computers had to be up to date;
%          \end{itemize}
%      \item Did some more administrating
%   \end{itemize}}

\tldatecventry{2021}{\href{https://github.com/fdavidcl/slicer-conv}{Slicer (convolutional)}}{Convolutional autoencoder model for complexity reduction}{}{}{Source code: \url{http://github.com/fdavidcl/slicer-conv}\newline%
Tensorflow implementation of a convolutional autoencoder which learns from labels with an SVM loss.}

\tldatecventry{2017}{\href{http://cometa.ujaen.es}{Cometa}}{The comprehensive multi-label data archive}{}{}{Source code: \url{https://github.com/fdavidcl/cometa}\newline%
Docker container that deploys an automatized web repository to prepare and host multi-label datasets.}

\tldatecventry{2016}{\href{http://ruta.software}{Ruta}}{Software for unsupervised deep architectures}{}{}{Source code: \url{https://github.com/fdavidcl/ruta}\newline%
R package for training unsupervised Deep Learning models.}

\tldatecventry{2014}{\href{http://fcharte.github.io/mldr}{mldr}}{R package for analyzing and manipulating multilabel datasets}{}{}{Source code: \url{https://github.com/fcharte/mldr}\newline%
R library for exploratory data analysis of multi-label datasets.}


%-------------------Interests Section-------------------------------------------------------------------

\section{Interests and communities}

% Format:  \cvitem{Hobby}{Description}
% Example: \cvitem{Gaming}{Computer Games}
% Example: \cvitem{Sport}{Golf, Tennis}
\cvitem{\textbf{Interests}}{Data science, free culture, (human and machine) languages, scientific dissemination}

\tlcventry{2014}{2020}{\href{https://libreim.github.io}{LibreIM}}{Student community dedicated to Mathematics and Computer Science}{Co-founder}{}{\begin{itemize}\item We organized \href{https://libreim.github.io/t/seminarios/}{regular talks \ExternalLink} for compsci \& math students, several of them given by myself\end{itemize}}
\tlcventry{2016}{2017}{\href{https://interferencias.tech}{Interferencias}}{Non-profit group interested in online rights and security}{Participant}{}{}

%-------------------References Section------------------------------------------------------------------

%% \section{References}

%% % Format:  \cvreferencecolumn{\cvreference{Name Surname}{Position}{Department}{Company}{City}{Email}{Home Phone}{Cell Phone}}{\cvreference{Name Surname}{Position}{Department}{Company}{City}{Email}{Home Phone}{Cell Phone}}
%% % Example: 
%% % \subsection{Simple Solutions}
%% % \cvreferencecolumn{\cvreference{John Doe}{Developer}{HR}{Simple Solutions}{MyCity}{john@email.com}{+12 (34) 567 8901}{+23 (45) 678 9012}}{\cvreference{Jane Doe}{Accountant}{HR}{Simple Solutions}{MyCity}{jane@email.com}{+34 (56) 789 0123}{+45 (67) 890 1234}}
%% % \subsection{Monster Inc}
%% % \cvreferencecolumn{\cvreference{Alice Doe}{Manager}{HR}{Monster Inc}{ThatCity}{alice@email.com}{+12 (34) 567 8901}{+23 (45) 678 9012}}{}

%% \subsection{Simple Solutions}
%% \cvreferencecolumn{\cvreference{John Doe}{Developer}{HR}{Simple Solutions}{MyCity}{john@email.com}{+12 (34) 567 8901}{+23 (45) 678 9012}}{\cvreference{Jane Doe}{Accountant}{HR}{Simple Solutions}{MyCity}{jane@email.com}{+34 (56) 789 0123}{+45 (67) 890 1234}} \subsection{Monster Inc}
%% \cvreferencecolumn{\cvreference{Alice Doe}{Manager}{HR}{Monster Inc}{ThatCity}{alice@email.com}{+12 (34) 567 8901}{+23 (45) 678 9012}}{}

\clearpage

%-------------------Appendix----------------------------------------------------------------------------
% This section is added to append any additional documents to the cv.
% The appended documents are added to the table of contents for easier navigation of the document.
% Usage: (section)
% \phantomsection
% \addcontentsline{toc}{section}{title}
% 
% Format: (subsection)
% \phantomsection\addcontentsline{toc}{subsection}{title}
% \includepdf[pages=-]{appendix/filename.pdf}
%
% Example:
% \phantomsection
% \addcontentsline{toc}{section}{Certificates}
%
% \phantomsection
% \addcontentsline{toc}{subsection}{Landscape}
% \includepdf[pages=-]{appendix/CertificateLandscape.pdf}
%
% \phantomsection
% \addcontentsline{toc}{subsection}{Portrait}
% \includepdf[pages=-]{appendix/CertificatePortrait.pdf}

%% \phantomsection
%% \addcontentsline{toc}{section}{Certificados}

%% \phantomsection
%% \addcontentsline{toc}{subsection}{Grado en Ingeniería Informática}
%% \includepdf[pages=-]{appendix/grado/resguardo_informatica}

%% \phantomsection
%% \addcontentsline{toc}{subsection}{Grado en Matemáticas}
%% \includepdf[pages=-]{appendix/grado/resguardo_matematicas}

%% \phantomsection
%% \addcontentsline{toc}{subsection}{International Summer School on Deep Learning}
%% \includepdf[pages=-]{appendix/cursos/deeplearn2017}

%% \phantomsection
%% \addcontentsline{toc}{subsection}{Aproximación práctica a la Ciencia de Datos y Big Data}
%% \includepdf[pages=-,landscape=true]{appendix/cursos/ciencia_datos}

%% \phantomsection
%% \addcontentsline{toc}{subsection}{Nuevas Tendencias en Ingeniería de Computadores}
%% \includepdf[pages=1,landscape=true]{appendix/cursos/tendencias-ic}

%% \phantomsection
%% \addcontentsline{toc}{subsection}{Estalmat}
%% \includepdf[pages=-,landscape=true]{appendix/cursos/estalmat}

%% \phantomsection
%% \addcontentsline{toc}{subsection}{Inglés C1}
%% \includepdf[pages=1]{appendix/idiomas/english_cae}

%% \phantomsection
%% \addcontentsline{toc}{subsection}{Francés B2}
%% \includepdf[pages=-,landscape=true]{appendix/idiomas/francais_b2}

%-------------------Cover letter------------------------------------------------------------------------

%% \input{coverletter.tex}                             % Include cover letter from coverletter.tex

%-------------------Document End------------------------------------------------------------------------

\end{document}

%% end of file `main.tex'.
